\documentclass[a4paper, 12pt]{article}

% PACKAGES
\usepackage[utf8]{inputenc}
\usepackage[T1]{fontenc}
\usepackage{graphicx}
\usepackage{amsmath}
\usepackage{amsfonts}
\usepackage{amssymb}
\usepackage[hungarian]{babel}
\usepackage{geometry}
\usepackage{xcolor}
\usepackage{hyperref}

% GEOMETRY
\geometry{
 a4paper,
 total={170mm,257mm},
 left=20mm,
 top=20mm,
}

% DOCUMENT
\begin{document}

% --- TITLE ---
\title{A 4D-k: Delegálás}
\author{Rick Dakan, Joseph Feller, and Anthropic}
\date{2025}
\maketitle

% --- CONTENT ---
\section*{A delegálás átgondolt döntések meghozatala arról, hogy melyik munka megfelelő az Ön számára, melyiket végezze az MI, vagy melyiket végezzék együtt, és hogyan osszák el ezeket a feladatokat.}
\textit{Mit próbálok csinálni? Mit csinálsz jól? Oké, itt a terv}

\begin{center}
\begin{tabular}{ccc}
\textbf{Problématudatosság} & \textbf{Platformtudatosság} & \textbf{Feladatdelegálás} \\
\end{tabular}
\end{center}

\begin{itemize}
    \item \textbf{Problématudatosság:} A célok és a munka természetének világos megértése, mielőtt bevonjuk az MI-t.
    \item \textbf{Platformtudatosság:} A különböző MI rendszerek képességeinek és korlátainak megértése.
    \item \textbf{Feladatdelegálás:} A munka átgondolt elosztása az emberek és az MI között, hogy kihasználják mindkettő erősségeit.
\end{itemize}

\textit{A hatékony delegálás mind szakterületi szakértelmet, mind az MI képességeinek megértését igényli.}

\vspace{\fill}
\begin{center}
    \small{Copyright 2025 Rick Dakan, Joseph Feller és Anthropic. Kiadva a CC BY-NC-SA 4.0 licenc alatt.}
\end{center}

\end{document}