\documentclass[a4paper, 12pt]{article}

% PACKAGES
\usepackage[utf8]{inputenc}
\usepackage[T1]{fontenc}
\usepackage{graphicx}
\usepackage{amsmath}
\usepackage{amsfonts}
\usepackage{amssymb}
\usepackage[hungarian]{babel}
\usepackage{geometry}
\usepackage{xcolor}
\usepackage{hyperref}

% GEOMETRY
\geometry{
 a4paper,
 total={170mm,257mm},
 left=20mm,
 top=20mm,
}

% DOCUMENT
\begin{document}

% --- TITLE ---
\title{MI Gondosság}
\author{Rick Dakan, Joseph Feller, and Anthropic}
\date{2025}
\maketitle

% --- CONTENT ---
\section*{Az MI Fluencia: Keretrendszer és Alapok kurzus fejlesztése}
Az MI Fluencia: Keretrendszer és Alapok kurzus fejlesztése során széleskörűen együttműködtünk az Anthropic Claude 3.7 nevű modelljével.

\section*{A kurzus alapját a következő anyagok képezték}
\begin{itemize}
    \item Az MI Fluencia Keretrendszer Gyakorlati Összefoglaló dokumentuma, melyet Rick Dakan (a Ringling Művészeti és Design Főiskolán) és Joseph Feller (a University College Cork-on) készített, valamint kapcsolódó munkadokumentumok és kutatási jegyzetek.
    \item Diabemutatók és előadás-átiratok több egyetemi kurzusról, vendégelőadásról és kutatási előadásról, melyeket Feller és/vagy Dakan tartott.
    \item Technikai/gyakorlati tartalmak, melyeket Maggie Vo és Drew Bent (Anthropic) biztosított.
\end{itemize}

\section*{A folyamat}
A folyamat során a Claude segítette a szerzők egyikét vagy többjét a szerkezeti fejlesztésben, az erőforrások és gyakorlatok tervezésében, valamint a tartalom vázlatának elkészítésében, kritizálásában, szerkesztésében és átírásában. Az emberi szerzők írták, tervezték, szerkesztették és biztosították a folyamatos jövőképet, szakértelmet, kritikai ítélőképességet és szakterületi tudást, és ők hozták meg minden végső döntést mind a tartalom, mind a megközelítés tekintetében.

\section*{Ellenőrzés}
Minden MI által generált és közösen létrehozott tartalmat az emberi szerzők alaposan ellenőriztek, szerkesztettek és gondoztak. A végleges anyagok pontosan tükrözik az emberi szerzők megértését, szakértelmét és tervezett pedagógiai megközelítését. Bár az MI segítsége nagyban hozzájárult ezeknek az anyagoknak az elkészítéséhez, a tartalomért a felelősséget az emberi szerzők viselik.

\section*{Átláthatóság}
Ez a nyilatkozat az MI Fluencia Keretrendszer által hirdetett átláthatóság szellemében készült, valamint elismerve az MI fejlődő szerepét az oktatási tartalomfejlesztésben és más kreatív és szellemi munkában.

\vspace{\fill}
\begin{center}
    \small{Copyright 2025 Rick Dakan, Joseph Feller és Anthropic. Kiadva a CC BY-NC-SA 4.0 licenc alatt.}
\end{center}

\end{document}