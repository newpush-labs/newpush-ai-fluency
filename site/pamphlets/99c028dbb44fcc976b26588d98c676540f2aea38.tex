\documentclass[a4paper, 12pt]{article}

% PACKAGES
\usepackage[utf8]{inputenc}
\usepackage[T1]{fontenc}
\usepackage{graphicx}
\usepackage{amsmath}
\usepackage{amsfonts}
\usepackage{amssymb}
\usepackage[hungarian]{babel}
\usepackage{geometry}
\usepackage{xcolor}
\usepackage{hyperref}

% GEOMETRY
\geometry{
 a4paper,
 total={170mm,257mm},
 left=20mm,
 top=20mm,
}

% DOCUMENT
\begin{document}

% --- TITLE ---
\title{A 4D-k: Megkülönböztetés}
\author{Rick Dakan, Joseph Feller, and Anthropic}
\date{2025}
\maketitle

% --- CONTENT ---
\section*{A megkülönböztetés az a képesség, hogy átgondoltan és kritikusan értékeljük, mit az MI termel, hogyan termeli, és hogyan viselkedik.}

\begin{center}
\begin{tabular}{ccc}
\textbf{Termék} & \textbf{Folyamat} & \textbf{Teljesítmény} \\
\textbf{Megkülönböztetés} & \textbf{Megkülönböztetés} & \textbf{Megkülönböztetés} \\
\end{tabular}
\end{center}

\begin{itemize}
    \item \textbf{Termék megkülönböztetés:} Annak értékelése, hogy mit az MI termel (pontosság, megfelelőség, koherencia, relevancia).
    \item \textbf{Folyamat megkülönböztetés:} Annak értékelése, hogyan jutott az MI a kimenetéhez, logikai hibákat, figyelmetlenségeket vagy nem megfelelő érvelési lépéseket keresve.
    \item \textbf{Teljesítmény megkülönböztetés:} Annak értékelése, hogyan viselkedik az MI az interakció során, figyelembe véve, hogy kommunikációs stílusa hatékony-e az Ön igényeinek.
\end{itemize}

\textit{A megkülönböztetés kéz a kézben működik a leírással egy folyamatos visszacsatolási hurokban.}

\vspace{\fill}
\begin{center}
    \small{Copyright 2025 Rick Dakan, Joseph Feller és Anthropic. Kiadva a CC BY-NC-SA 4.0 licenc alatt.}
\end{center}

\end{document}