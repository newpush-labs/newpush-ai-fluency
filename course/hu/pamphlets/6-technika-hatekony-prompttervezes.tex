\documentclass[a4paper, 11pt]{article}

% PACKAGES
\usepackage[utf8]{inputenc}
\usepackage[T1]{fontenc}
\usepackage{graphicx}
\usepackage{amsmath}
\usepackage{amsfonts}
\usepackage{amssymb}
\usepackage[hungarian]{babel}
\usepackage{geometry}
\usepackage{xcolor}
\usepackage{hyperref}
\usepackage{titlesec}
\usepackage{enumitem}
\usepackage{fancyhdr}
\usepackage{tcolorbox}
\usepackage{listings}
\usepackage[scaled=0.9]{roboto-mono}

% GEOMETRY - Match the MkDocs content width
\geometry{
 a4paper,
 total={160mm,240mm},
 left=25mm,
 top=25mm,
 bottom=25mm,
}

% COLORS - Match MkDocs theme colors
\definecolor{primary}{HTML}{1967C0}
\definecolor{secondary}{HTML}{1976D2}
\definecolor{accent}{HTML}{2980B9}
\definecolor{textcolor}{HTML}{2C3E50}
\definecolor{lightgray}{HTML}{F8F9FA}
\definecolor{codebg}{HTML}{F1F3F4}
\definecolor{codetext}{HTML}{D73502}

% FONT SETUP - Use Roboto to match MkDocs
\renewcommand{\familydefault}{\sfdefault}
\usepackage[default]{roboto}

% TYPOGRAPHY - Match MkDocs hierarchy
\titleformat{\section}
  {\fontsize{18}{22}\bfseries\color{primary}}
  {\thesection}{1em}{}
  [\vspace{0.5em}{\color{lightgray}\titlerule[2pt]}]

\titleformat{\subsection}
  {\fontsize{15}{18}\bfseries\color{secondary}}
  {\thesubsection}{1em}{}
  [\vspace{0.3em}{\color{lightgray}\titlerule[1pt]}]

\titleformat{\subsubsection}
  {\fontsize{12}{15}\bfseries\color{textcolor}}
  {\thesubsubsection}{1em}{}

% SPACING - Match MkDocs spacing
\setlength{\parindent}{0pt}
\setlength{\parskip}{1.25em}
\renewcommand{\baselinestretch}{1.6}

% LIST STYLING - Match MkDocs custom lists
\setlist[itemize]{
  leftmargin=1.5em,
  itemsep=0.75em,
  parsep=0pt,
  topsep=0.5em
}

\setlist[enumerate]{
  leftmargin=2em,
  itemsep=0.75em,
  parsep=0pt,
  topsep=0.5em
}

% HYPERLINKS - Match MkDocs link styling
\hypersetup{
  colorlinks=true,
  linkcolor=accent,
  urlcolor=accent,
  citecolor=accent,
  filecolor=accent
}

% HEADER/FOOTER - Clean professional look
\pagestyle{fancy}
\fancyhf{}
\fancyhead[L]{\small\color{textcolor}6 technika a hatékony prompttervezéshez}
\fancyhead[R]{\small\color{textcolor}Az MI Fluencia Keretrendszer}
\fancyfoot[C]{\small\color{textcolor}\thepage}
\renewcommand{\headrulewidth}{0.5pt}
\renewcommand{\footrulewidth}{0pt}

% TITLE STYLING - Match MkDocs h1 styling
\makeatletter
\renewcommand{\maketitle}{
  \begin{center}
    {\fontsize{22}{26}\bfseries\color{primary}\@title}
    \vspace{0.5em}
    {\color{lightgray}\titlerule[3pt]}
    \vspace{1.5em}
  \end{center}
}
\makeatother

% BLOCKQUOTE STYLING - Match MkDocs enhanced blockquotes
\newenvironment{customquote}
  {\begin{tcolorbox}[
    enhanced,
    left=4mm,
    right=4mm,
    top=4mm,
    bottom=4mm,
    colback=lightgray,
    colframe=accent,
    leftrule=4mm,
    rightrule=0mm,
    toprule=0mm,
    bottomrule=0mm,
    arc=2mm,
    outer arc=2mm
  ]}
  {\end{tcolorbox}}

% CODE STYLING - Match MkDocs code blocks
\lstset{
  basicstyle=\small\ttfamily\color{textcolor},
  backgroundcolor=\color{codebg},
  frame=single,
  frameround=tttt,
  framerule=0.5pt,
  rulecolor=\color{lightgray},
  breaklines=true,
  breakatwhitespace=true,
  showspaces=false,
  showstringspaces=false,
  numbers=none,
  xleftmargin=8pt,
  xrightmargin=8pt,
  aboveskip=8pt,
  belowskip=8pt
}

% INLINE CODE - Match MkDocs inline code
\newcommand{\inlinecode}[1]{%
  \colorbox{codebg}{\texttt{\color{codetext}#1}}%
}

% DOCUMENT
\begin{document}

% --- TITLE ---
\title{6 technika a hatékony prompttervezéshez}
\author{Rick Dakan, Joseph Feller, and Anthropic}
\date{2025}
\maketitle

% --- CONTENT ---
\section*{1. Adjon kontextust}
\begin{tabular}{p{0.45\textwidth} p{0.45\textwidth}}
\textbf{Előtte} & \textbf{Utána} \\
\parbox[t]{0.45\textwidth}{Mesélj a klímaváltozásról.} & \parbox[t]{0.45\textwidth}{Magyarázza el a klímaváltozás három fő hatását a mezőgazdaságra a trópusi régiókban, az elmúlt évtizedből vett példákkal.} \\
\end{tabular}
\textit{Legyen konkrét abban, amit szeretne, adjon meg részleteket a hatókörről, a földrajzi területről, az időkeretről és más releváns paraméterekről.}

\section*{2. Mutasson példákat arra, hogy mi a "jó"}
\begin{tabular}{p{0.45\textwidth} p{0.45\textwidth}}
\textbf{Előtte} & \textbf{Utána} \\
\parbox[t]{0.45\textwidth}{Kérem, alakítsa át ezt a technikai kijelentést közérthetővé: "A platform végponttól végpontig terjedő titkosítási protokollokat alkalmaz az adatintegritás védelme érdekében."} & \parbox[t]{0.45\textwidth}{Itt van két példa arra, hogyan lehet a technikai zsargont közérthető nyelvre átalakítani: \newline 1. "A kvantumalgoritmus kvadratikus gyorsulást mutat." - "Az új módszer nagyjából kétszer olyan gyorsan oldja meg a problémákat, mint a korábbi módszerek." \newline 2. "A felület intuitív tervezési paradigmákat használ." - "A design könnyen érthető és használható." \newline Most kérem, alakítsa át ezt a technikai kijelentést közérthető nyelvre: "A platform végponttól végpontig terjedő titkosítási protokollokat alkalmaz az adatintegritás védelme érdekében."} \\
\end{tabular}
\textit{A példák bemutatása segít az MI-nek megérteni a mintát, stílust vagy formátumot, amit keres, sokkal világosabban, mint a leírások önmagukban.}

\section*{3. Adjon meg kimeneti korlátokat}
\begin{tabular}{p{0.45\textwidth} p{0.45\textwidth}}
\textbf{Előtte} & \textbf{Utána} \\
\parbox[t]{0.45\textwidth}{Tervezzen nekem egy személyes művészeti portfólió weboldalt.} & \parbox[t]{0.45\textwidth}{Hozzon létre egy letisztult, modern, egyoldalas portfólió weboldalt ezekkel a szakaszokkal: Hős, Rólam, Készségek, Portfólió/Projektek, Tapasztalat és Kapcsolat. Tegye a navigációs menüt ragadóssá és reszponzívvá, hamburger menüvel mobilon. Használjon naplemente színpalettát és adjon hozzá egy sötét/világos mód kapcsolót.} \\
\end{tabular}
\textit{Legyen konkrét abban, amit szeretne, adjon meg részleteket a hatókörről, a földrajzi területről, az időkeretről és más releváns paraméterekről.}

\section*{4. Bontsa a komplex feladatokat lépésekre}
\begin{tabular}{p{0.45\textwidth} p{0.45\textwidth}}
\textbf{Előtte} & \textbf{Utána} \\
\parbox[t]{0.45\textwidth}{Elemezze ezt a negyedéves értékesítési adatot.} & \parbox[t]{0.45\textwidth}{Szeretném elemezni ezt a negyedéves értékesítési adatot. Kérem, közelítse meg ezt a következőképpen: \newline 1. A legjobban teljesítő termékek azonosítása \newline 2. A jelenlegi negyedév összehasonlítása az előző negyedévvel \newline 3. A szokatlan minták vagy trendek kiemelése \newline 4. Lehetséges okok javaslata ezekre a trendekre} \\
\end{tabular}
\textit{A komplex feladatok világos lépésekre bontása irányítja az MI érvelési folyamatát és alapos, módszeres válaszokat biztosít.}

\section*{5. Kérje meg, hogy először gondolkodjon}
\textit{Válaszadás előtt kérem, gondolja át alaposan ezt a problémát. Vegye figyelembe a különböző tényezőket, a lehetséges korlátokat és a különböző megközelítéseket, mielőtt a legjobb megoldást javasolná.}
\textit{Ha teret ad az MI-nek a gondolkodásra a válaszadás előtt, az ösztönzi a mélyebb, átfogóbb válaszokat.}

\section*{6. Határozza meg az MI szerepét}
\textit{Kérem, magyarázza el, hogyan alakulnak ki a szivárványok egy tapasztalt természettudomány-tanár szemszögéből, aki egy okos, 10 éves gyereknek beszél, akit érdekel a tudomány.}
\textit{Az MI szerepének, hangnemének vagy stílusának meghatározása segít formálni a megközelítését, hogy illeszkedjen az Ön specifikus igényeihez és közönségéhez.}

\section*{Titkos fegyver: Kérjen segítséget az MI-től a promptoláshoz}
\textit{Megpróbálom rávenni Önt, Claude, hogy segítsen nekem [cél]. Nem vagyok benne biztos, hogyan fogalmazzam meg a kérésemet, hogy a legjobb eredményt kapjam. Segítene nekem egy hatékony promptot készíteni ehhez?}
\textit{Ha nem biztos benne, hogyan kérjen valamit, az MI segíthet javítani a promptján. Ez talán a leghatékonyabb technika mind közül!}

\vspace{\fill}
\begin{center}
    \small{Copyright 2025 Rick Dakan, Joseph Feller és Anthropic. Kiadva a CC BY-NC-SA 4.0 licenc alatt.}
\end{center}

\end{document}