\documentclass[a4paper, 11pt]{article}

% PACKAGES
\usepackage[utf8]{inputenc}
\usepackage[T1]{fontenc}
\usepackage{graphicx}
\usepackage{amsmath}
\usepackage{amsfonts}
\usepackage{amssymb}
\usepackage[hungarian]{babel}
\usepackage{geometry}
\usepackage{xcolor}
\usepackage{hyperref}
\usepackage{titlesec}
\usepackage{enumitem}
\usepackage{fancyhdr}
\usepackage{tcolorbox}
\usepackage{listings}
\usepackage[scaled=0.9]{roboto-mono}

% GEOMETRY - Match the MkDocs content width
\geometry{
 a4paper,
 total={160mm,240mm},
 left=25mm,
 top=25mm,
 bottom=25mm,
}

% COLORS - Match MkDocs theme colors
\definecolor{primary}{HTML}{1967C0}
\definecolor{secondary}{HTML}{1976D2}
\definecolor{accent}{HTML}{2980B9}
\definecolor{textcolor}{HTML}{2C3E50}
\definecolor{lightgray}{HTML}{F8F9FA}
\definecolor{codebg}{HTML}{F1F3F4}
\definecolor{codetext}{HTML}{D73502}

% FONT SETUP - Use Roboto to match MkDocs
\renewcommand{\familydefault}{\sfdefault}
\usepackage[default]{roboto}

% TYPOGRAPHY - Match MkDocs hierarchy
\titleformat{\section}
  {\fontsize{18}{22}\bfseries\color{primary}}
  {\thesection}{1em}{}
  [\vspace{0.5em}{\color{lightgray}\titlerule[2pt]}]

\titleformat{\subsection}
  {\fontsize{15}{18}\bfseries\color{secondary}}
  {\thesubsection}{1em}{}
  [\vspace{0.3em}{\color{lightgray}\titlerule[1pt]}]

\titleformat{\subsubsection}
  {\fontsize{12}{15}\bfseries\color{textcolor}}
  {\thesubsubsection}{1em}{}

% SPACING - Match MkDocs spacing
\setlength{\parindent}{0pt}
\setlength{\parskip}{1.25em}
\renewcommand{\baselinestretch}{1.6}

% LIST STYLING - Match MkDocs custom lists
\setlist[itemize]{
  leftmargin=1.5em,
  itemsep=0.75em,
  parsep=0pt,
  topsep=0.5em
}

\setlist[enumerate]{
  leftmargin=2em,
  itemsep=0.75em,
  parsep=0pt,
  topsep=0.5em
}

% HYPERLINKS - Match MkDocs link styling
\hypersetup{
  colorlinks=true,
  linkcolor=accent,
  urlcolor=accent,
  citecolor=accent,
  filecolor=accent
}

% HEADER/FOOTER - Clean professional look
\pagestyle{fancy}
\fancyhf{}
\fancyhead[L]{\small\color{textcolor}MI Gondosság}
\fancyhead[R]{\small\color{textcolor}Az MI Fluencia Keretrendszer}
\fancyfoot[C]{\small\color{textcolor}\thepage}
\renewcommand{\headrulewidth}{0.5pt}
\renewcommand{\footrulewidth}{0pt}

% TITLE STYLING - Match MkDocs h1 styling
\makeatletter
\renewcommand{\maketitle}{
  \begin{center}
    {\fontsize{22}{26}\bfseries\color{primary}MI Gondosság}
    \vspace{0.5em}
    {\color{lightgray}\titlerule[3pt]}
    \vspace{1.5em}
  \end{center}
}
\makeatother

% DOCUMENT
\begin{document}

% --- TITLE ---
\title{MI Gondosság}
\author{Rick Dakan, Joseph Feller, and Anthropic}
\date{2025}
\maketitle

% --- CONTENT ---
\section*{Az MI Fluencia: Keretrendszer és Alapok kurzus fejlesztése}
Az MI Fluencia: Keretrendszer és Alapok kurzus fejlesztése során széleskörűen együttműködtünk az Anthropic Claude 3.7 nevű modelljével.

\section*{A kurzus alapját a következő anyagok képezték}
\begin{itemize}
    \item Az MI Fluencia Keretrendszer Gyakorlati Összefoglaló dokumentuma, melyet Rick Dakan (a Ringling Művészeti és Design Főiskolán) és Joseph Feller (a University College Cork-on) készített, valamint kapcsolódó munkadokumentumok és kutatási jegyzetek.
    \item Diabemutatók és előadás-átiratok több egyetemi kurzusról, vendégelőadásról és kutatási előadásról, melyeket Feller és/vagy Dakan tartott.
    \item Technikai/gyakorlati tartalmak, melyeket Maggie Vo és Drew Bent (Anthropic) biztosított.
\end{itemize}

\section*{A folyamat}
A folyamat során a Claude segítette a szerzők egyikét vagy többjét a szerkezeti fejlesztésben, az erőforrások és gyakorlatok tervezésében, valamint a tartalom vázlatának elkészítésében, kritizálásában, szerkesztésében és átírásában. Az emberi szerzők írták, tervezték, szerkesztették és biztosították a folyamatos jövőképet, szakértelmet, kritikai ítélőképességet és szakterületi tudást, és ők hozták meg minden végső döntést mind a tartalom, mind a megközelítés tekintetében.

\section*{Ellenőrzés}
Minden MI által generált és közösen létrehozott tartalmat az emberi szerzők alaposan ellenőriztek, szerkesztettek és gondoztak. A végleges anyagok pontosan tükrözik az emberi szerzők megértését, szakértelmét és tervezett pedagógiai megközelítését. Bár az MI segítsége nagyban hozzájárult ezeknek az anyagoknak az elkészítéséhez, a tartalomért a felelősséget az emberi szerzők viselik.

\section*{Átláthatóság}
Ez a nyilatkozat az MI Fluencia Keretrendszer által hirdetett átláthatóság szellemében készült, valamint elismerve az MI fejlődő szerepét az oktatási tartalomfejlesztésben és más kreatív és szellemi munkában.

\vspace{\fill}
\begin{center}
    \small{Copyright 2025 Rick Dakan, Joseph Feller és Anthropic. Kiadva a CC BY-NC-SA 4.0 licenc alatt.\\A magyar nyelvű változatot készítette Rácz László és a NewPush Europe Kft.}
\end{center}

\end{document}