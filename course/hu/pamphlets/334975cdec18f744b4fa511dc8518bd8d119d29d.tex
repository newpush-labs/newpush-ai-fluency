\documentclass[a4paper, 12pt]{article}

% PACKAGES
\usepackage[utf8]{inputenc}
\usepackage[T1]{fontenc}
\usepackage{graphicx}
\usepackage{amsmath}
\usepackage{amsfonts}
\usepackage{amssymb}
\usepackage[hungarian]{babel}
\usepackage{geometry}
\usepackage{xcolor}
\usepackage{hyperref}

% GEOMETRY
\geometry{
 a4paper,
 total={170mm,257mm},
 left=20mm,
 top=20mm,
}

% DOCUMENT
\begin{document}

% --- TITLE ---
\title{Az MI Fluencia Keretrendszer}
\author{Rick Dakan, Joseph Feller, and Anthropic}
\date{2025}
\maketitle

% --- CONTENT ---
\section*{Négy egymással összefüggő kompetencia, amelyek biztosítják, hogy az MI-vel való interakcióink hatékonyak, eredményesek, etikusak és biztonságosak legyenek.}

\subsection*{A 4D Keretrendszer}
\begin{description}
    \item[Delegálás] Célok kitűzése és annak eldöntése, hogy mikor és hogyan vonjuk be az MI-t.
    \item[Leírás] A célok hatékony leírása a hasznos MI viselkedések és kimenetek előidézéséhez.
    \item[Megkülönböztetés] Az MI kimeneteinek és viselkedésének hasznosságának pontos felmérése.
    \item[Gondosság] Felelősségvállalás azért, amit az MI-vel teszünk és ahogyan tesszük.
\end{description}

\section*{Kulcsfogalmak}
\begin{description}
    \item[Az MI interakció három módja]
    \begin{itemize}
        \item \textbf{Automatizálás:} Az MI konkrét feladatokat hajt végre emberi utasítások alapján.
        \item \textbf{Bővítés:} Az emberek és az MI gondolkodó partnerként működnek együtt a feladatok elvégzésében.
        \item \textbf{Ügynökség:} Az emberek úgy konfigurálják az MI-t, hogy önállóan végezzen jövőbeli feladatokat a nevükben.
    \end{itemize}
    \item[MI Fluencia] Az MI Fluencia azt jelenti, hogy az MI rendszerekkel olyan módon lépünk kapcsolatba, amely hatékony, eredményes, etikus és biztonságos. A delegálás, leírás, megkülönböztetés és gondosság kompetenciák egymással összefüggő készségek, ismeretek, belátások, értékek gyűjteményei.
\end{description}

\vspace{\fill}
\begin{center}
    \small{Copyright 2025 Rick Dakan, Joseph Feller és Anthropic. Kiadva a CC BY-NC-SA 4.0 licenc alatt.}
\end{center}

\end{document}