\documentclass[a4paper, 11pt]{article}

% PACKAGES
\usepackage[utf8]{inputenc}
\usepackage[T1]{fontenc}
\usepackage{graphicx}
\usepackage{amsmath}
\usepackage{amsfonts}
\usepackage{amssymb}
\usepackage[hungarian]{babel}
\usepackage{geometry}
\usepackage{xcolor}
\usepackage{hyperref}
\usepackage{titlesec}
\usepackage{enumitem}
\usepackage{fancyhdr}
\usepackage{tcolorbox}
\usepackage{listings}
\usepackage[scaled=0.9]{roboto-mono}

% GEOMETRY - Match the MkDocs content width
\geometry{
 a4paper,
 total={160mm,240mm},
 left=25mm,
 top=25mm,
 bottom=25mm,
}

% COLORS - Match MkDocs theme colors
\definecolor{primary}{HTML}{1967C0}
\definecolor{secondary}{HTML}{1976D2}
\definecolor{accent}{HTML}{2980B9}
\definecolor{textcolor}{HTML}{2C3E50}
\definecolor{lightgray}{HTML}{F8F9FA}
\definecolor{codebg}{HTML}{F1F3F4}
\definecolor{codetext}{HTML}{D73502}

% FONT SETUP - Use Roboto to match MkDocs
\renewcommand{\familydefault}{\sfdefault}
\usepackage[default]{roboto}

% TYPOGRAPHY - Match MkDocs hierarchy
\titleformat{\section}
  {\fontsize{18}{22}\bfseries\color{primary}}
  {\thesection}{1em}{}
  [\vspace{0.5em}{\color{lightgray}\titlerule[2pt]}]

\titleformat{\subsection}
  {\fontsize{15}{18}\bfseries\color{secondary}}
  {\thesubsection}{1em}{}
  [\vspace{0.3em}{\color{lightgray}\titlerule[1pt]}]

\titleformat{\subsubsection}
  {\fontsize{12}{15}\bfseries\color{textcolor}}
  {\thesubsubsection}{1em}{}

% SPACING - Match MkDocs spacing
\setlength{\parindent}{0pt}
\setlength{\parskip}{1.25em}
\renewcommand{\baselinestretch}{1.6}

% LIST STYLING - Match MkDocs custom lists
\setlist[itemize]{
  leftmargin=1.5em,
  itemsep=0.75em,
  parsep=0pt,
  topsep=0.5em
}

\setlist[enumerate]{
  leftmargin=2em,
  itemsep=0.75em,
  parsep=0pt,
  topsep=0.5em
}

% HYPERLINKS - Match MkDocs link styling
\hypersetup{
  colorlinks=true,
  linkcolor=accent,
  urlcolor=accent,
  citecolor=accent,
  filecolor=accent
}

% HEADER/FOOTER - Clean professional look
\pagestyle{fancy}
\fancyhf{}
\fancyhead[L]{\small\color{textcolor}A generatív MI megértése}
\fancyhead[R]{\small\color{textcolor}Az MI Fluencia Keretrendszer}
\fancyfoot[C]{\small\color{textcolor}\thepage}
\renewcommand{\headrulewidth}{0.5pt}
\renewcommand{\footrulewidth}{0pt}

% TITLE STYLING - Match MkDocs h1 styling
\makeatletter
\renewcommand{\maketitle}{
  \begin{center}
    {\fontsize{22}{26}\bfseries\color{primary}\@title}
    \vspace{0.5em}
    {\color{lightgray}\titlerule[3pt]}
    \vspace{1.5em}
  \end{center}
}
\makeatother

% BLOCKQUOTE STYLING - Match MkDocs enhanced blockquotes
\newenvironment{customquote}
  {\begin{tcolorbox}[
    enhanced,
    left=4mm,
    right=4mm,
    top=4mm,
    bottom=4mm,
    colback=lightgray,
    colframe=accent,
    leftrule=4mm,
    rightrule=0mm,
    toprule=0mm,
    bottomrule=0mm,
    arc=2mm,
    outer arc=2mm
  ]}
  {\end{tcolorbox}}

% CODE STYLING - Match MkDocs code blocks
\lstset{
  basicstyle=\small\ttfamily\color{textcolor},
  backgroundcolor=\color{codebg},
  frame=single,
  frameround=tttt,
  framerule=0.5pt,
  rulecolor=\color{lightgray},
  breaklines=true,
  breakatwhitespace=true,
  showspaces=false,
  showstringspaces=false,
  numbers=none,
  xleftmargin=8pt,
  xrightmargin=8pt,
  aboveskip=8pt,
  belowskip=8pt
}

% INLINE CODE - Match MkDocs inline code
\newcommand{\inlinecode}[1]{%
  \colorbox{codebg}{\texttt{\color{codetext}#1}}%
}

% DOCUMENT
\begin{document}

% --- TITLE ---
\title{A generatív MI megértése}
\author{Rick Dakan, Joseph Feller, and Anthropic}
\date{2025}
\maketitle

% --- CONTENT ---
\section*{Mi a generatív MI?}
A generatív MI olyan mesterséges intelligencia rendszereket jelent, amelyek új tartalmat tudnak létrehozni ahelyett, hogy csak a meglévő adatokat elemeznék.

\subsection*{Hagyományos MI vs. Generatív MI}
\begin{tabular}{ll}
\textbf{Hagyományos MI} & \textbf{Generatív MI} \\
E-maileket spamként vagy nem spamként osztályoz & Teljesen új e-mailt tud írni neked \\
\end{tabular}

\section*{Három pillér, amely lehetővé tette}
\begin{itemize}
    \item \textbf{Algoritmusok:} A transzformátor architektúra (2017) forradalmasította a hosszú szöveges szövegek feldolgozását.
    \item \textbf{Adatrobbanás:} A digitális adatok robbanásszerű növekedése (webhelyek, kódtárak és egyéb szövegek) nyersanyagot biztosított ezeknek a rendszereknek a betanításához.
    \item \textbf{Számítási teljesítmény:} A számítási teljesítmény masszív növekedése (chipek, mint a GPU-k) tette lehetővé ezeknek a modelleknek a betanítását az összes adaton.
\end{itemize}

\section*{Hogyan működik}
\begin{description}
    \item[Előképzés] A modellek több milliárd szöveges példát elemeznek, megtanulva előre jelezni, mi következik.
    \item[Finomhangolás] A modelleket finomítják, hogy kövessék az utasításokat, segítőkészek legyenek, és elkerüljék a káros tartalmakat.
    \item[Telepítés] A felhasználók promptokat adnak meg, és a modell a promptok és a betanítás során tanult minták alapján generál válaszokat.
\end{description}

\section*{Főbb képességek és jelenlegi korlátok}
\begin{tabular}{p{0.45\textwidth} p{0.45\textwidth}}
\textbf{Főbb képességek} & \textbf{Jelenlegi korlátok} \\
Sokoldalú nyelvi készség & Tudásbázis-határidő \\
Általános célú képességek & Lehetséges pontatlanságok ("hallucinációk") \\
Példából való tanulás & Kontextusablak-korlát \\
\parbox[t]{0.45\textwidth}{Külső eszközökhöz és adatokhoz való csatlakozás} & \parbox[t]{0.45\textwidth}{Kihívások a komplex érveléssel és matematikával} \\
\end{tabular}

\vspace{\fill}
\begin{center}
    \small{Copyright 2025 Rick Dakan, Joseph Feller és Anthropic. Kiadva a CC BY-NC-SA 4.0 licenc alatt.\\A magyar nyelvű változatot készítette Rácz László és a NewPush Europe Kft.}
\end{center}

\end{document}