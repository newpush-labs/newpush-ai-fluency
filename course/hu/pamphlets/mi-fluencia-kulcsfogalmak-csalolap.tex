\documentclass[a4paper, 11pt]{article}

% PACKAGES
\usepackage[utf8]{inputenc}
\usepackage[T1]{fontenc}
\usepackage{graphicx}
\usepackage{amsmath}
\usepackage{amsfonts}
\usepackage{amssymb}
\usepackage[hungarian]{babel}
\usepackage{geometry}
\usepackage{xcolor}
\usepackage{hyperref}
\usepackage{titlesec}
\usepackage{enumitem}
\usepackage{fancyhdr}
\usepackage{tcolorbox}
\usepackage{listings}
\usepackage[scaled=0.9]{roboto-mono}
\usepackage{multicol}

% GEOMETRY - Match the MkDocs content width
\geometry{
 a4paper,
 total={160mm,240mm},
 left=20mm,
 top=20mm,
 bottom=20mm,
}

% COLORS - Match MkDocs theme colors
\definecolor{primary}{HTML}{1967C0}
\definecolor{secondary}{HTML}{1976D2}
\definecolor{accent}{HTML}{2980B9}
\definecolor{textcolor}{HTML}{2C3E50}
\definecolor{lightgray}{HTML}{F8F9FA}
\definecolor{codebg}{HTML}{F1F3F4}
\definecolor{codetext}{HTML}{D73502}

% FONT SETUP - Use Roboto to match MkDocs
\renewcommand{\familydefault}{\sfdefault}
\usepackage[default]{roboto}

% TYPOGRAPHY - Match MkDocs hierarchy
\titleformat{\section}
  {\fontsize{16}{20}\bfseries\color{primary}}
  {\thesection}{1em}{}
  [\vspace{0.3em}{\color{lightgray}\titlerule[1.5pt]}]

\titleformat{\subsection}
  {\fontsize{13}{16}\bfseries\color{secondary}}
  {\thesubsection}{1em}{}
  [\vspace{0.2em}{\color{lightgray}\titlerule[0.5pt]}]

\titleformat{\subsubsection}
  {\fontsize{11}{14}\bfseries\color{textcolor}}
  {\thesubsubsection}{1em}{}

% SPACING - Match MkDocs spacing
\setlength{\parindent}{0pt}
\setlength{\parskip}{0.8em}
\renewcommand{\baselinestretch}{1.3}

% LIST STYLING - Match MkDocs custom lists
\setlist[itemize]{
  leftmargin=1.2em,
  itemsep=0.4em,
  parsep=0pt,
  topsep=0.3em
}

\setlist[enumerate]{
  leftmargin=1.5em,
  itemsep=0.4em,
  parsep=0pt,
  topsep=0.3em
}

% HYPERLINKS - Match MkDocs link styling
\hypersetup{
  colorlinks=true,
  linkcolor=accent,
  urlcolor=accent,
  citecolor=accent,
  filecolor=accent
}

% HEADER/FOOTER - Clean professional look
\pagestyle{fancy}
\fancyhf{}
\fancyhead[L]{\small\color{textcolor}MI Fluencia: Kulcsfogalmak Csalólap}
\fancyhead[R]{\small\color{textcolor}ANTHROPIC}
\fancyfoot[C]{\small\color{textcolor}\thepage}
\renewcommand{\headrulewidth}{0.5pt}
\renewcommand{\footrulewidth}{0pt}

% TITLE STYLING - Match MkDocs h1 styling
\makeatletter
\renewcommand{\maketitle}{
  \begin{center}
    {\fontsize{20}{24}\bfseries\color{primary}\@title}
    \vspace{0.3em}
    {\color{lightgray}\titlerule[2pt]}
    \vspace{1em}
  \end{center}
}
\makeatother

% BLOCKQUOTE STYLING - Match MkDocs enhanced blockquotes
\newenvironment{customquote}
  {\begin{tcolorbox}[
    left=3mm,
    right=3mm,
    top=3mm,
    bottom=3mm,
    colback=lightgray,
    colframe=accent,
    leftrule=3mm,
    rightrule=0mm,
    toprule=0mm,
    bottomrule=0mm,
    arc=1mm,
    outer arc=1mm
  ]}
  {\end{tcolorbox}}

% DOCUMENT
\begin{document}

% --- TITLE ---
\title{MI Fluencia: Kulcsfogalmak Csalólap}
\author{Rick Dakan, Joseph Feller és Anthropic}
\date{2025}
\maketitle

\begin{multicols}{2}

\section{Alapvető MI Fluencia Keretrendszer Fogalmak}

\subsection{MI Fluencia}
Az a képesség, hogy hatékonyan, eredményesen, etikusan és biztonságosan dolgozzunk az MI rendszerekkel. Magában foglalja a gyakorlati készségeket, ismereteket, belátásokat és értékeket.

\subsection{A 4D-k}
Az MI Fluencia négy alapkompetenciája: Delegálás, Leírás, Megkülönböztetés és Gondosság.

\subsection{Delegálás}
Annak eldöntése, hogy mely munkát végezzék emberek, melyiket az MI, és hogyan osszák el a feladatokat közöttük.

\subsubsection{Komponensek:}
\begin{itemize}
\item \textbf{Problématudatosság:} A célok és a munka természetének világos megértése
\item \textbf{Platformtudatosság:} A különböző MI rendszerek képességeinek megértése  
\item \textbf{Feladatdelegálás:} A munka átgondolt elosztása emberek és MI között
\end{itemize}

\subsection{Leírás}
Hatékony kommunikáció az MI rendszerekkel. Magában foglalja a kimenetek egyértelmű meghatározását és az MI folyamatok irányítását.

\subsubsection{Komponensek:}
\begin{itemize}
\item \textbf{Termékleírás:} Mit szeretnél a kimenetek, formátum, közönség és stílus tekintetében
\item \textbf{Folyamatleírás:} Hogyan közelítse meg az MI a kérésedet
\item \textbf{Teljesítményleírás:} Az MI viselkedésének meghatározása az együttműködés során
\end{itemize}

\subsection{Megkülönböztetés}
Az MI kimeneteinek, folyamatainak és viselkedésének átgondolt és kritikus értékelése.

\subsubsection{Komponensek:}
\begin{itemize}
\item \textbf{Termék megkülönböztetés:} Mit produkál az MI (pontosság, megfelelőség)
\item \textbf{Folyamat megkülönböztetés:} Hogyan jutott el a kimenetéhez
\item \textbf{Teljesítmény megkülönböztetés:} Hogyan viselkedik az MI az interakció során
\end{itemize}

\subsection{Gondosság}
Az MI felelősségteljes és etikus használata. Magában foglalja az átgondolt döntéseket és a felelősségvállalást.

\subsubsection{Komponensek:}
\begin{itemize}
\item \textbf{Létrehozási gondosság:} Mely MI rendszereket használsz és hogyan
\item \textbf{Átláthatósági gondosság:} Őszintének lenni az MI szerepéről
\item \textbf{Telepítési gondosság:} Felelősséget vállalni a kimenetek ellenőrzéséért
\end{itemize}

\section{Ember-MI interakciós módok}

\subsection{Automatizálás}
Az MI konkrét feladatokat hajt végre konkrét emberi utasítások alapján. Az ember meghatározza, mit kell tenni, és az MI végrehajtja azt.

\subsection{Bővítés}
Az emberek és az MI gondolkodó partnerként működnek együtt. Iteratív oda-vissza folyamat, ahol mindkettő hozzájárul az eredményhez.

\subsection{Ügynökség}
Az emberek úgy konfigurálják az MI-t, hogy önállóan dolgozzon a nevükben. Az ember határozza meg az MI tudását és viselkedési mintáit.

\section{MI technikai koncepciók}

\subsection{Generatív MI}
MI rendszerek, amelyek új tartalmat tudnak létrehozni, nem csak a meglévő adatokat elemzik.

\subsection{Nagy nyelvi modellek (LLM-ek)}
Hatalmas mennyiségű szöveges adaton betanított generatív MI rendszerek, hogy megértsék és generálják az emberi nyelvet.

\subsection{Claude}
Az Anthropic nagy nyelvi modelljeinek családja.

\subsection{Érvelési vagy gondolkodási modellek}
Az MI modellek típusai, amelyeket kifejezetten arra terveztek, hogy lépésről lépésre gondolkodjanak a komplex problémákon.

\subsection{Hőmérséklet}
Egy beállítás, amely szabályozza, hogy egy MI válaszai mennyire véletlenszerűek. Magasabb hőmérséklet → kreatívabb kimenetek. Alacsonyabb hőmérséklet → kiszámíthatóbb válaszok.

\subsection{Paraméterek}
Az MI modellen belüli matematikai értékek, amelyek meghatározzák, hogyan dolgozza fel az információkat. A modern LLM-ek milliárdnyi paramétert tartalmaznak.

\subsection{Neurális hálózatok}
Összekapcsolt csomópontokból álló számítástechnikai rendszerek, amelyek rétegekbe vannak szervezve, és a betanítás során mintákat tanulnak.

\subsection{Transzformátor architektúra}
Az áttörést jelentő MI tervezés 2017-ből, amely lehetővé teszi az LLM-ek számára, hogy párhuzamosan dolgozzák fel a szövegszekvenciákat.

\subsection{Skálázási törvények}
Ahogy az MI modellek nagyobbak lettek, teljesítményük következetes minták szerint javult. Bizonyos küszöbökön teljesen új képességek jelenhetnek meg.

\subsection{Előképzés}
A kezdeti betanítási fázis, ahol az MI modellek hatalmas mennyiségű szöveges adatból tanulnak mintákat.

\subsection{Finomhangolás}
További betanítás az előképzés után, ahol a modellek megtanulnak utasításokat követni és segítőkész válaszokat adni.

\subsection{Visszakereséssel bővített generálás (RAG)}
Egy technika, amely külső tudásforrásokhoz kapcsolja az MI modelleket a pontosság javítása érdekében.

\subsection{Torzítás}
Az MI kimeneteiben megjelenő szisztematikus minták, amelyek tisztességtelenül előnyben részesítenek bizonyos csoportokat.

\section{Prompttervezési koncepciók}

\subsection{Prompt}
Az MI modellnek adott bemenet, beleértve az utasításokat és a megosztott dokumentumokat.

\subsection{Prompttervezés}
A hatékony promptok tervezésének gyakorlata az MI rendszerek számára a kívánt kimenetek előállítása érdekében.

\subsection{Gondolatmenet-promptolás}
Az MI ösztönzése arra, hogy lépésről lépésre dolgozzon végig egy problémát, a komplex feladatokat kisebb lépésekre bontva.

\subsection{Néhány példás tanulás (n-shot prompting)}
Az MI tanítása a kívánt bemenet-kimenet minta példáinak bemutatásával. Az "N" a megadott példák számára utal.

\subsection{Szerep- vagy perszóna-meghatározás}
Egy adott karakter, szakértelem vagy kommunikációs stílus meghatározása az MI számára. Lehet általános ("UX tervezési szakértő") vagy konkrét ("Richard Feynman").

\subsection{Kontextusablak}
Az az információs mennyiség, amelyet egy MI egyszerre figyelembe vehet, beleértve a beszélgetési előzményeket.

\subsection{Kimeneti korlátok/formázás}
A promptban egyértelműen meghatározni a kívánt formátumot, hosszúságot, szerkezetet az MI válaszában.

\subsection{Gondolkodj először megközelítés}
Kifejezetten arra kérni az MI-t, hogy dolgozza végig az érvelési folyamatát, mielőtt végleges választ adna.

\subsection{Hallucináció}
Olyan hiba, amikor az MI magabiztosan állít valamit, ami hihetőnek hangzik, de valójában helytelen.

\subsection{Tudásbázis-határidő}
Az a pont, amely után egy MI modellnek nincs beépített tudása a világról, attól függően, hogy mikor tanították be.

\end{multicols}

\vfill
\begin{center}
\small
Copyright 2025 Rick Dakan, Joseph Feller és Anthropic. Kiadva a CC BY-NC-SA 4.0 licenc alatt.
\end{center}

\end{document}