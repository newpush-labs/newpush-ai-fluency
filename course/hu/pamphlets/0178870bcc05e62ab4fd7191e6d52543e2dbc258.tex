\documentclass[a4paper, 12pt]{article}

% PACKAGES
\usepackage[utf8]{inputenc}
\usepackage[T1]{fontenc}
\usepackage{graphicx}
\usepackage{amsmath}
\usepackage{amsfonts}
\usepackage{amssymb}
\usepackage[hungarian]{babel}
\usepackage{geometry}
\usepackage{xcolor}
\usepackage{hyperref}

% GEOMETRY
\geometry{
 a4paper,
 total={170mm,257mm},
 left=20mm,
 top=20mm,
}

% DOCUMENT
\begin{document}

% --- TITLE ---
\title{A 4D-k: Leírás}
\author{Rick Dakan, Joseph Feller, and Anthropic}
\date{2025}
\maketitle

% --- CONTENT ---
\section*{A leírás az a képesség, hogy úgy kommunikáljunk az MI-vel, hogy az produktív együttműködési környezetet teremtsen.}
\textit{Hé! Találd ki, mire gondolok? Sóhaj. Az első dolog, amit tennünk kell... Legyél szókratészi tutor.}

\begin{center}
\begin{tabular}{ccc}
\textbf{Termék} & \textbf{Folyamat} & \textbf{Teljesítmény} \\
\textbf{Leírás} & \textbf{Leírás} & \textbf{Leírás} \\
\end{tabular}
\end{center}

\begin{itemize}
    \item \textbf{Termékleírás:} Annak meghatározása, hogy mit szeretne a kimenetek, a formátum, a közönség, és a stílus tekintetében.
    \item \textbf{Folyamatleírás:} Annak meghatározása, hogy az MI hogyan közelítse meg a kérését, például lépésről lépésre történő utasítások megadásával az MI számára.
    \item \textbf{Teljesítményleírás:} Az MI rendszer viselkedésének meghatározása az együttműködés során, például hogy tömör vagy részletes, kihívó vagy támogató legyen.
\end{itemize}

\textit{A tiszta kommunikáció az MI rendszerekkel időt takarít meg és jobb eredményekhez vezet.}

\vspace{\fill}
\begin{center}
    \small{Copyright 2025 Rick Dakan, Joseph Feller és Anthropic. Kiadva a CC BY-NC-SA 4.0 licenc alatt.}
\end{center}

\end{document}