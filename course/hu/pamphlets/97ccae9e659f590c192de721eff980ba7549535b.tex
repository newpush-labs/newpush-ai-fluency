\documentclass[a4paper, 12pt]{article}

% PACKAGES
\usepackage[utf8]{inputenc}
\usepackage[T1]{fontenc}
\usepackage{graphicx}
\usepackage{amsmath}
\usepackage{amsfonts}
\usepackage{amssymb}
\usepackage[hungarian]{babel}
\usepackage{geometry}
\usepackage{xcolor}
\usepackage{hyperref}

% GEOMETRY
\geometry{
 a4paper,
 total={170mm,257mm},
 left=20mm,
 top=20mm,
}

% DOCUMENT
\begin{document}

% --- TITLE ---
\title{A 4D-k: Gondosság}
\author{Rick Dakan, Joseph Feller, and Anthropic}
\date{2025}
\maketitle

% --- CONTENT ---
\section*{A gondosság felelősségvállalás azért, amit az MI-vel teszünk és ahogyan tesszük.}
\textit{“Ebben a dokumentumban, jót tettünk. Claude 3.7 ...”}

\begin{center}
\begin{tabular}{ccc}
\textbf{Létrehozás} & \textbf{Átláthatóság} & \textbf{Telepítés} \\
\textbf{Gondosság} & \textbf{Gondosság} & \textbf{Gondosság} \\
\end{tabular}
\end{center}

\begin{itemize}
    \item \textbf{Létrehozási gondosság:} Átgondoltan választja ki, hogy melyik MI rendszert használja és hogyan lép kapcsolatba velük.
    \item \textbf{Átláthatósági gondosság:} Átláthatóan kommunikálja az MI szerepét a munkájában mindenkivel, akinek tudnia kell róla.
    \item \textbf{Telepítési gondosság:} Felelősséget vállal az Ön által használt vagy megosztott kimenetek ellenőrzéséért és igazolásáért.
\end{itemize}

\textit{Különböző kontextusok (személyes, tudományos, szakmai) eltérő elvárásokat támaszthatnak a közzététel és az ellenőrzés tekintetében.}

\vspace{\fill}
\begin{center}
    \small{Copyright 2025 Rick Dakan, Joseph Feller és Anthropic. Kiadva a CC BY-NC-SA 4.0 licenc alatt.}
\end{center}

\end{document}