\documentclass[a4paper, 12pt]{article}

% PACKAGES
\usepackage[utf8]{inputenc}
\usepackage[T1]{fontenc}
\usepackage{graphicx}
\usepackage{amsmath}
\usepackage{amsfonts}
\usepackage{amssymb}
\usepackage[hungarian]{babel}
\usepackage{geometry}
\usepackage{xcolor}
\usepackage{hyperref}

% GEOMETRY
\geometry{
 a4paper,
 total={170mm,257mm},
 left=20mm,
 top=20mm,
}

% DOCUMENT
\begin{document}

% --- TITLE ---
\title{A generatív MI megértése}
\author{Rick Dakan, Joseph Feller, and Anthropic}
\date{2025}
\maketitle

% --- CONTENT ---
\section*{Mi a generatív MI?}
A generatív MI olyan mesterséges intelligencia rendszereket jelent, amelyek új tartalmat tudnak létrehozni ahelyett, hogy csak a meglévő adatokat elemeznék.

\subsection*{Hagyományos MI vs. Generatív MI}
\begin{tabular}{ll}
\textbf{Hagyományos MI} & \textbf{Generatív MI} \\
E-maileket spamként vagy nem spamként osztályoz & Teljesen új e-mailt tud írni Önnek \\
\end{tabular}

\section*{Három pillér, amely lehetővé tette}
\begin{itemize}
    \item \textbf{Algoritmusok:} A transzformátor architektúra (2017) forradalmasította a hosszú szöveges szövegek feldolgozását.
    \item \textbf{Adatrobbanás:} A digitális adatok robbanásszerű növekedése (webhelyek, kódtárak és egyéb szövegek) nyersanyagot biztosított ezeknek a rendszereknek a betanításához.
    \item \textbf{Számítási teljesítmény:} A számítási teljesítmény masszív növekedése (chipek, mint a GPU-k) tette lehetővé ezeknek a modelleknek a betanítását az összes adaton.
\end{itemize}

\section*{Hogyan működik}
\begin{description}
    \item[Előképzés] A modellek több milliárd szöveges példát elemeznek, megtanulva előre jelezni, mi következik.
    \item[Finomhangolás] A modelleket finomítják, hogy kövessék az utasításokat, segítőkészek legyenek, és elkerüljék a káros tartalmakat.
    \item[Telepítés] A felhasználók promptokat adnak meg, és a modell a promptok és a betanítás során tanult minták alapján generál válaszokat.
\end{description}

\section*{Főbb képességek és jelenlegi korlátok}
\begin{tabular}{p{0.45\textwidth} p{0.45\textwidth}}
\textbf{Főbb képességek} & \textbf{Jelenlegi korlátok} \\
Sokoldalú nyelvi készség & Tudásbázis-határidő \\
Általános célú képességek & Lehetséges pontatlanságok ("hallucinációk") \\
Példából való tanulás & Kontextusablak-korlát \\
\parbox[t]{0.45\textwidth}{Külső eszközökhöz és adatokhoz való csatlakozás} & \parbox[t]{0.45\textwidth}{Kihívások a komplex érveléssel és matematikával} \\
\end{tabular}

\vspace{\fill}
\begin{center}
    \small{Copyright 2025 Rick Dakan, Joseph Feller és Anthropic. Kiadva a CC BY-NC-SA 4.0 licenc alatt.}
\end{center}

\end{document}